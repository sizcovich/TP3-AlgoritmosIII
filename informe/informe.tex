\nonstopmode
\documentclass[10pt, a4paper]{article}
%\usepackage{subfig}

\parindent=20pt
\parskip=8pt
\usepackage[width=15.5cm, left=3cm, top=2.5cm, height= 24.5cm]{geometry}
\usepackage[spanish]{babel}
\usepackage[utf8]{inputenc}
\usepackage{fancyhdr}
\usepackage{multirow}
\usepackage{rotating}
\usepackage{indentfirst}
\usepackage{latexsym}
\usepackage{caratula}
\usepackage{gnuplottex}
\usepackage{epsfig}
\usepackage{lastpage}
\usepackage{amsfonts}
\usepackage{listings}
\usepackage[export]{adjustbox}
\usepackage{pdfpages}
\lstset{language=C}
\usepackage[ruled,vlined,linesnumbered]{algorithm2e}
\usepackage{graphicx}
\usepackage{float}
\usepackage{color}

\graphicspath{{imgs/}}



% Acomodo fancyhdr.
\pagestyle{fancy}
\thispagestyle{fancy}
\addtolength{\headheight}{1pt}
\lhead{Algoritmos y Estructuras de Datos III}
\rhead{TP3}
\cfoot{\thepage /\pageref{LastPage}}
\renewcommand{\footrulewidth}{0.4pt}
\renewcommand{\thesubsubsection}{\thesubsection.\alph{subsubsection}}


\author{Algoritmos y Estructuras de Datos III, DC, UBA.}
\date{}
\title{}

\begin{document}
	
\thispagestyle{empty}
\materia{Algoritmos y Estructuras de Datos III}
\submateria{Trabajo Pr\'actico N$^{\circ}$3}
\titulo{}
\integrante{Izcovich, Sabrina}{550/11}{sizcovich@gmail.com}
\integrante{Garcia Marset, Matias}{356/11}{matiasgarciamarset@gmail.com}
\integrante{Orellana, Ignacio}{229/11}{nacho@foxdev.com.ar}
\integrante{Vita, Sebastián}{149/11}{sebastian\_vita@yahoo.com.ar}

\maketitle

\tableofcontents
\newpage

\section{Introducci\'on}
El siguiente trabajo práctico consiste en un estudio de distintas formas de resolver un mismo problema algorítmico a partir de diversas técnicas.

El problema se basa en hallar la clique\footnote{Dado un grafo simple $G$ = ($V;E$), un subconjunto de vértices de $G$ es una clique sí y sólo sí éste induce un subgrafo completo de $G$.} de máxima frontera\footnote{Definimos la frontera de una clique como $\delta(K) = \left \{ vw \in E / v \in K \land w \in V \setminus K \right \}$.} (CMF) de un grafo. Es decir que, dado un grafo $G$, debemos encontrar una clique $K$ cuya frontera $\delta(K)$ tenga cardinalidad máxima. Para resolver el problema, supondremos que los grafos son simples, con lo cual no tienen bucles ni ejes repetidos.


\textbf{Formato de entrada:}
La entrada contiene varias instancias del problema. Cada instancia representa un grafo $G$ y comienza con una línea con dos valores enteros $n$ (cantidad de vértices) y $m$ (cantidad de aristas). A continuación, le siguen $m$ líneas, cada una determinando una arista del grafo, cuyo formato es el siguiente:
$$v_{1}\ v_{2}$$
donde $v_{1}$ y $v_{2}$ son los extremos de la arista representada (numerados de 1 a $n$).

\textbf{Formato de salida:} 
La salida contiene una línea por cada instancia de entrada, con el siguiente formato:
$$F\ k\ v_{1}\ v_{2}\ ...\ v_{k}$$
donde $F$ es el cardinal de la frontera de la clique dada como solución, $k$ es el tamaño de la misma y $v_{1}$, ..., $v_{k}$ son los vértices que la conforman.\newline


Para resolver el problema presentado, debimos diseñar e implementar los siguientes tipos de algoritmos:

\begin{itemize}
 \item Un algoritmo exacto.
 \item Al menos una heurística constructiva golosa.
 \item Al menos una heurística de búsqueda local.
 \item Al menos un algoritmo que use la metaheurística \textit{Búsqueda Tabú}\footnote{Fred Glover. Tabu Search - Part 1. ORSA Journal on Computing 1 (3), pp: 190 \& Part 2. ORSA Journal on Computing 2 (1), pp: 4.}
\end{itemize}

El trabajo se organizó de la siguiente forma:
\begin{itemize}
\item El \textbf{ejercicio 1} fue utilizado para describir situaciones de la vida real que pudieran ser modeladas utilizando CMF.
\item En el \textbf{ejercicio 2}, diseñamos un algoritmo exacto para CMF explicando detalladamente su implementación. Luego, calculamos el orden de complejidad temporal del peor caso tomando a $n$ como la cantidad de nodos del grafo de entrada. Finalmente, explicitamos la experimentación realizada capaz de comprobar la performance del algoritmo realizado comparando su tiempo de ejecución en función del tamaño del parámetro de entrada.
\item El \textbf{ejercicio 3} consistió en presentar distintos tipos de heurísticas, constructivas golosas y de búsqueda local, que resolvieran el problema descrito como también algoritmos que utilizaran la metaheurística Búsqueda Tabú. Cada algoritmo debió ser explicado detalladamente y su orden de complejidad temporal debió ser explicitado al igual que instancias de CMF para las cuales los métodos no proporcionan una solución óptima. Por último, se realizó la experimentación necesaria para observar la performance de cada algoritmo comparando la calidad de las soluciones obtenidas y los tiempos de ejecución de los mismos en función de la entrada.
\item Finalmente, en el \textbf{ejercicio 4}, evaluamos los algoritmos propuestos sobre un nuevo conjunto de instancias una vez elegidos los mejores valores de configuración para cada heurística.
\newpage

\section{Descripción de situaciones reales}
En este ejercicio, nos limitamos a pensar situaciones de la vida real que pudieran ser modeladas con $cliques\ de\ máxima\ frontera$.
\begin{itemize}
\item Juanita es una apasionada de conocer calles nuevas y desea mudarse a algún barrio de una ciudad que le permita seguir con su pasión. Pero Juanita trabaja en el campo, o sea que fuera de cualquier ciudad, así que se le ocurre seleccionar su próxima vivienda de acuerdo a la cantidad de calles que va a poder conocer para llegar al campo. Podemos modelar el problema como los nodos representantes de las esquinas y las aristas las interconexiones entre éstas. Llamamos ciudad a todas las esquinas que se conectan todas con todas, luego, a una clique y campo a todo lo que se encuentra entre una ciudad y otra. Por lo tanto, se desea encontrar una ciudad con mayor frontera para que la cantidad de posibles calles nuevas a recorrer fuera de la ciudad sea la mayor.

\item Se vienen las elecciones y como todos los años, el futuro candidato a presidente Josecito quiere invertir en repavimentación. Esta vez quiere elegir las esquinas interconectadas entre sí que tengan la mayor cantidad de calles en su frontera. Esto se debe a que las máquinas de repavimentación, que se encuentran en cada esquina de la zona de trabajo, necesitan estar conectadas entre sí dado que, cuando se quedan sin algún material necesitan llegar rápidamente a otra de las máquinas para terminar velozmente el trabajo. Si modelamos a las esquinas y a las aristas como calles, Josecito debe encontrar la clique de mayor frontera. 
\end{itemize}

\newpage

\section{Detalles de implementación}
Para la realización de las Heurísticas Golosa, Búsqueda Local y Búsqueda Tabú, utilizamos la clase \textit{Grafo}. Dicha clase fue establecida a partir del Taller de Heurísticas\footnote{http://www.dc.uba.ar/materias/aed3/2013/2c/laboratorio/taller-heuristicas/view} realizado en los laboratorios. En ella, se hace uso de un vector de adyacencia y una matriz de adyacencia, dependiendo del caso. Por ejemplo, para saber si dos nodos son vecinos, resulta más simple utilizar una matriz de adyacencia pues simplemente debe verificarse una única posición. Por otro lado, para obtener la totalidad de los vecinos de un nodo, es conveniente usar un vector de adyacencia pues no es necesario recorrer ninguna estructura para obtener dicho resultado. 

Analicemos la complejidad temporal de la clase \textit{Grafo}:
\begin{itemize}
\item \textbf{Definición de grafo:}
\begin{itemize}
\item $vecinos(nodos, vuint()) \rightarrow \mathcal{O}(n)$ pues $vecinos$ es $std::vector< vuint > vecinos$ y por cada elemento debe realizarse un $push\_back$($\mathcal{O}(1)$ amortizado)\footnote{http://www.cplusplus.com/reference/vector/vector/push\_back/}, siendo $\mathcal{O}(n)*\mathcal{O}(1) = \mathcal{O}(n)$ el resultado final.
\item $adyacencia(nodos, std::vector<bool>(nodos, false)) \rightarrow \mathcal{O}(n^2)$ pues $adyacencia$ es $std::vector<std::vector<bool> > adyacencia$ y realiza $n\ push\_back$ por cada elemento, luego, su complejidad temporal resulta $\mathcal{O}(n)*\mathcal{O}(1)*\mathcal{O}(n)*\mathcal{O}(1) = \mathcal{O}(n^2)$.
\item $cnodos(nodos) \rightarrow \mathcal{O}(1)$ pues $nodos$ es $uint\ cnodos$.
\end{itemize}
\item \textbf{Funciones públicas:}
\begin{itemize}
\item $nodos$ devuelve $cnodos$, luego su complejidad es $\mathcal{O}(1)$.
\item $sonVecinos$ accede a una posición de $adyacencia$, luego, su complejidad temporal es $\mathcal{O}(1)*\mathcal{O}(1) = \mathcal{O}(1)$\footnote{http://www.cplusplus.com/reference/vector/vector/operator[]/}.
\item $agregarArista$ realiza dos $push\_back (\mathcal{O}(1) + \mathcal{O}(1) = \mathcal{O}(1)$) al vector $vecinos$ y, finalmente, le asigna un valor booleano a dos posiciones de $adyacencia$, siendo esto $\mathcal{O}(1)$ amortizado\footnote{http://www.cplusplus.com/reference/vector/vector/assign/}.
\item $vecindad$ retorna el valor de un elemento del vector $vecinos$, luego, la complejidad resulta $\mathcal{O}(1)$.
\item $nodoDeMayorGrado$ recorre la totalidad de los nodos ($cnodos$) del grafo (en $\mathcal{O}(n)$). En cada iteración, compara el tamaño (utilizando la función size() ($\mathcal{O}(1)$)\footnote{http://www.cplusplus.com/reference/vector/vector/size/}) de los vecinos de dos nodos y asigna un valor a un uint ($\mathcal{O}(1)$) dependiendo del resultado de la comparación. Luego, la complejidad temporal de dicha función resulta $\mathcal{O}(n)$.
\item $frontera$ recorre los nodos de una clique del grafo (cuyo tamaño máximo es $n$) y le asigna a un $uint$ el tamaño (utilizando size()) de un acceso a $vecinos$ ($\mathcal{O}(1)$). Luego, la complejidad de la función resulta $\mathcal{O}(n)*\mathcal{O}(1) = \mathcal{O}(n)$.
\end{itemize}
\end{itemize}

\newpage
\section{Algoritmo Exacto}
\subsection{Algoritmo Implementado}

Para la resolución exacta del problema a resolver, realizamos un algoritmo que recorriera todas las posibles cliques del grafo de entrada. Esto se debió a que no hallamos ninguna caracterización que nos permitiera reducir el espacio de búsqueda de cliques para que el algoritmo ignorara una cantidad asintótica de ellas. Luego, nuestra implementación genera todas las soluciones siguiendo un orden lexicográfico para evitar repetir ejecuciones. Por ejemplo, la clique $k_{2}$ formada por los nodos \{1,2\} es la misma que se forma al elegir los nodos \{2,1\}. De esta forma, al representar los nodos como números y formar cliques respetando su orden lexicográfico, nos ahorramos pasar por soluciones que ya formamos antes. %para mi hay que decir que chequea si ya lo ejecuto, porque el orden lexicografico solo me dice cual va a verificar primero...

Luego, nuestro algoritmo comienza generando todas las cliques que contienen al nodo de menor valor numérico que no haya sido utilizado y, en cada paso, va a seleccionar una serie de candidatos a extender la clique actual hasta no encontrar más y volver hacia atras, quitando el último nodo que se agregó a la clique, y eligiendo el próximo nodo que sea candidato si es que lo hubiera. El orden lexicografico es garantizado al momento de elegir los candidatos, ya que solo elegira aquellos candidatos cuyo numero de nodo sea mayor al numero de nodo del ultimo nodo que fue agregado a la clique.

A continuación, se expone el pseudocódigo del algoritmo:

Variables globales: \textbf{Conj(nodo)} $cliqueMaxima$, \textbf{Entero} $fronteraMaxima$.\\

\begin{algorithm}[H]
	\SetAlgoLined
	\caption{Exacto}
	\KwIn{\textbf{Grafo} $g$} %esto hay que ponerlo bien
	\KwOut{\textbf{Conj(nodo)} $clique$}

	\textbf{Conj(nodo)} $clique = \emptyset$\\
	\textbf{Conj(nodo)} $candidatos = $ nodos($g$)\\

	Backtracking($clique$, $candidatos$)\\

	\textbf{devuelvo} $cliqueMaxima$\\

\end{algorithm}

\begin{algorithm}[H]
	\SetAlgoLined
	\caption{Backtracking}
	\KwIn{\textbf{Conj(nodo)} $clique$, \textbf{Conj(nodo)} $candidatos$, \textbf{Grafo} $g$} %esto hay que ponerlo bien
	\KwOut{\textbf{Conj(nodo)} $clique$}

	\For{$c \in candidatos$}{
		\If{\textbf{frontera}($ clique \cup \{c\} $) es mayor a $fronteraMaxima$ }{
			$cliqueMaxima = clique \cup \{c\} $\\
			$fronteraMaxima = $frontera($clique \cup \{c\}$)\\
		}
		\If{\textbf{frontera}($ clique \cup \{c\} $) $>$ \textbf{frontera}($clique$)}{
			Backtracking($ clique \cup \{c\} $, calcularCandidatos($ clique \cup \{c\} $), $g$, $c$)\\
		}
	}
\end{algorithm}

\begin{algorithm}[H]
	\SetAlgoLined
	\caption{frontera}
	\KwIn{\textbf{Conj(nodo)} $clique$, \textbf{Grafo} $g$} %esto hay que ponerlo bien
	\KwOut{\textbf{Entero} $res$}

	$res = 0$\\
	\For{\textbf{nodo} $v \in nodos(g)$}{
		$res = res + (\#adyacentes(v) - (\#clique - 1))$\\
	}

	\textbf{devuelvo} $res$\\

\end{algorithm}

\begin{algorithm}[H]
	\SetAlgoLined
	\caption{calcularCandidatos}
	\KwIn{\textbf{Conj(nodo)} $clique$, \textbf{Grafo} $g$, \textbf{Nodo} $ultimoNodoAgregado$} %esto hay que ponerlo bien
	\KwOut{\textbf{Conj(nodo)} $candidatos$}

	nodo $w$ = un elemento de $clique$\\

	$candidatos = adyacentes(w)$\\
	\For{\textbf{todo nodo} $v \in nodos(g), v \neq w$}{
		$candidatos$ = interseccion entre $adyacentes(v)$ y $candidatos$\\
	}

	quitar de $candidatos$ todos los nodos menores numericamente a $ultimoNodoAgregado$

	\textbf{devuelvo} $candidatos$\\

\end{algorithm}

En este caso el programa comienza con la funcion Exacto, esta es la encargada de crear los valores inciales con los cuales se va a llamara a la funcion backtracking, luego esta hace las llamdas recursivas para ir creando las diferentes cliques y sus candidatos teniendo en cuenta el orden lexicografico y en cada paso chequeando si las cliques que considera son mejores que la mejor encontrada actualmente. \\
La funcion $calcularCandidatos$ busca aquellos nodos que no son los de la clique actual pero que estan conectados con todos ellos, eso se logra calculando la interseccion entre los nodos adyacentes a los nodos de la clique. \\
La funcion $frontera$ devuelve el valor numerico de la frontera de una clique, como tenemos por pre-condicion que todo conjunto de nodos que reciba como parametro la funcion es una clique, entonces calcular la frontera se puede hacer restandole a la cantidad de nodos adyacentes de cada nodo la cantidad de nodos de la clique menos 1. 

\subsection{Análisis de complejidad}

El analisis de complejidad del algoritmo posee una cota superior definida por la funcion recursiva de backtracking, esta funcion se pueden analizar calcular el arbol de soluciones del problema en cuestion. Como buscamos cliques y puede haber cliques de tamaño 1 hasta tamaño $n$ inclusive, entonces el arbol de soluciones va a tener un tamaño de 
\newpage

\section{Heurística Golosa}
\subsection{Explicación del algoritmo realizado}
Para resolver el algoritmo presentado anteriormente con una técnica golosa, decidimos utilizar el procedimiento que se presenta a continuación:\newline
\newline
\begin{algorithm}[H]
    \SetAlgoLined
    \caption{HeurísticaGolosa}
    \KwIn{\textbf{Grafo} $g$}
    \KwOut{\textbf{Conj(nodos)} $clique$}
	Entero $nodoDeMayorGrado$ = nodoDeMasGrado($g$)\\
	Grafo $cliqueHastaAhora$ = $\emptyset$\\
	agregar($v$, $cliqueHastaAhora$)\\
	\ForAll{$u \in$ adyacentes(nodoDeMayorGrado, nodos($g$))}{
		\If{forma una clique($g$, agregar($u$, $cliqueHastaAhora$)) $\land$ frontera($g$, agregar($u$, $cliqueHastaAhora$) $>$ frontera($cliqueHastaAhora$)}{
		agregar($u$, $cliqueHastaAhora$)}}
\textbf{devolver} $cliqueHastaAhora$
\end{algorithm}

donde $nodoDeMasGrado$ consiste en una función que toma el nodo del grafo cuyo grado es el mayor, $frontera$ es una función que calcula la frontera de un conjunto de nodos dentro de un grafo y $adyacentes$ consiste en una lista de nodos adyacentes a un determinado nodo.\newline



\subsection{Complejidad Temporal}
Veamos cuál es la complejidad temporal del algoritmo realizado:
\begin{itemize}
\item En primer lugar, el algoritmo utiliza la función $esClique$ para verificar si un conjunto de nodos de un grafo forman una clique. Dicha función consiste en dos ciclos, uno dentro de otro, que iteran cantidad de la clique veces, lo que resulta $\mathcal{O}(n^2)$ como peor caso. Dentro de los ciclos mencionados, se realizan dos comparaciones, donde una de ellas utiliza la función $sonVecinos$ ($\mathcal{O}(1)$). Luego, la complejidad temporal de la función $esClique$ resulta $\mathcal{O}(n^2)*\mathcal{O}(1)$ = $\mathcal{O}(n^2)$.

\item Por otro lado, la función $greedySearch$ crea un vector inicializado en 1 de tamaño $nodoDeMayorGrado$ ($\mathcal{O}(n)$) y un entero al que se le asigna el valor de la frontera de $cliqueHastaAhora$. Dicho valor se calcula con $frontera$ en $\mathcal{O}(n)$. Luego, se ejecuta un ciclo con la cantidad de nodos ($\mathcal{O}(n)$) en el que se realiza un $push\_back$ ($\mathcal{O}(1)$ amortizado\footnote{http://www.cplusplus.com/reference/vector/vector/push\_back/}) y se verifica que el vector $cliqueHastaAhora$ sea una clique con la función $esClique$ ($\mathcal{O}(n^2)$). Si dicha verificación resulta afirmativa, se compara la frontera de $cliqueHastaAhora$ con un entero ($\mathcal{O}(n)$) y luego se le asigna la nueva frontera a $fronteraHastaAhora$ (con la función $frontera$ en $\mathcal{O}(n)$). Caso contrario, se retira el último elemento agregado a la clique con $pop\_back()$ ($\mathcal{O}(1)$\footnote{http://www.cplusplus.com/reference/vector/vector/pop\_back/}).\newline
\newline
Luego, la complejidad temporal de la función $greedySearch$ resulta $2^*\mathcal{O}(n)+\mathcal{O}(n)^*\mathcal{O}(n)^*\mathcal{O}(n^2)$ = $\mathcal{O}(n^4)$.

\end{itemize}
\subsection{Instancias problemáticas}
Hay que decir que es asíntico, que se rompe muy fácil, como por ejemplo cuando la clique de maxima frontera no contiene al nodo de mayor grado.
\subsection{Experimentación}




\newpage

\section{Heurística de Búsqueda Local}
\subsection{Explicación del algoritmo realizado}
Los algoritmos de búsqueda local parten de una solución inicial $S_{0}$ y en cada paso intentan mejorarla. Para esto, se calculan todas las posibles variaciones de $S_{0}$ que forman una solución al problema. Al conjunto de todas estas se lo llama vecindad. 
Estos algoritmos se ejecutan siempre y cuando exista una solución, perteneciente a la vecindad, que sea mejor a la que ya teníamos. \newline \newline
Para nuestro ejercicio, utilizamos como $S_{0}$ al nodo de mayor frontera. Dicha elección se debió a que un nodo forma una clique, con lo cual es una posible solución a nuestro problema. Además, sabemos que este existe para cualquier grafo ya que la menor cantidad de nodos que se pueden ingresar en nuestro programa es uno. 
\newline Por otro lado, decidimos definir como vecindad a todas las posibles cliques que difieran en a lo sumo un nodo con nuestro $S_{0}$. Esta decision la tomamos para que nuestra clique no quede fija alrededor del nodo inicial. Para evitarnos tener que almacenar todas las posibles cliques para posteriormente elegir la mejor, decidimos separar la vecindad en tres subconjuntos. Estos se formaron con las cliques que cumplen lo siguiente:
\begin{itemize}
\item \textbf{Tiene todos los nodos de $S_{0}$ salvo 1:} \newline En lugar de calcular todas las posibles cliques de este subconjunto, decidimos quedarnos únicamente con la clique que se obtiene de quitarle el nodo de menor grado a $S_{0}$. Esto se debe a que la frontera de una clique se puede calcular con la siguiente formula:\newline
Sea $S_{0}$ = ($V$,$E$) y n = $|$$V$$|$
\begin{equation}
  \delta(S_{0}) = \sum_{v \in V}^{} d(v) - n*(n-1)
\end{equation}
Ahora, si quitamos un nodo $v'$ de $S_{0}$, obtenemos lo siguiente:
\begin{equation}
  \delta(S'_{0}) = \sum_{v \in V/v'}^{} d(v) - (n-1)*(n-2)
\end{equation}
Como nosotros queremos encontrar el mayor \delta$(S'_{0})$, debemos quedarnos con el que tiene la sumatoria de mayor valor ya que (n-1)*(n-2) es igual para todas las cliques. Con lo cual, el nodo que debemos eliminar para poder maximizar la sumatoria, es el de menor grado.

El pseudo codigo de este subconjunto de la vecindad es:\newline
\begin{algorithm}[H]
    \SetAlgoLined
    \caption{quitarNodo}
    \KwIn{\textbf{Grafo} $grafo$, \textbf{Conj(Entero)} $clique$}
    \KwOut{\textbf{par(Entero,Conj(Entero))} $res$}
	
    \textbf{Entero} $minimo$ = $|$vecindad(0)$|$ \\	
    \textbf{Entero} $minimoNodo = 0$ \\
    \ForAll{$v \in$ $|clique|$}{
        $aux$ = $|$vecindad($v$)$|$ \\
		\If{$aux < minimo$}{
            $minimo$ = $aux$ \\
			$minimoNodo$ = $v$
	 		}}
    
    \textbf{Conj(Entero)} $cliqueQuitando$\\

    \ForAll{$v \in$ $|clique|$}{
		\If{$v \neq minimoNodo$}{
            agregar($cliqueQuitando$, $v$)
	 		}}
    
    \textbf{Entero} $fronteraRes$ = frontera($grafo$, $cliqueQuitando$)\\
    $res$ = hacerPar($fronteraRes$, $cliqueQuitando$)\\
    \textbf{devolver} $res$ \\
\end{algorithm}

Donde $frontera$ calcula la frontera del subgrafo pasado por parametro, $agregar$ inserta un elemento en un arreglo, $vecindad$ nos devuelve todos los nodos adyacentes al nodo pasado por parametro y $hacerPar$ genera un par con lo dos elementos pasados por parametro. \newline


\item \textbf{Tiene todos los nodos de $S_{0}$ más uno que no pertenecía a el:} \newline
Para poder obtener la clique de frontera máxima de este subconjunto, buscamos todos los nodos del grafo que pueden formar una clique con $S_{0}$. Para esto, contamos cuantos nodos de la clique se conectan a cada nodo del grafo. Si un nodo es alcanzado por todos los que pertenecen a $S_{0}$, entonces ese nodo forma una clique. Una vez que tenemos a todos los posibles candidatos, calculamos la frontera de cada uno y nos quedamos con la mayor. \newline

\begin{algorithm}[H]
    \SetAlgoLined
    \caption{agregarNodo}
    \KwIn{\textbf{Grafo} $grafo$, \textbf{Conj(Entero)} $clique$}
    \KwOut{\textbf{par(Entero,Conj(Entero))} $res$}
	
    \textbf{Conj(Entero)} $bucket[$Nodos($grafo$)$] = 0$  \\
	
    \ForAll{$u \in$ Nodos($clique$)}{
	\ForAll{$v \in$ vecindad($u$)}{
		$bucket[v]$++
	 		}}

    \ForAll{$u \in$ Nodos($clique$)}{
	$bucket[u] = 0$
	}
	
    \textbf{Conj(Entero)} $posibleClique$ \\
    \ForAll{$u \in$ Nodos($grafo$)}{
	\If{$bucket[u] = |clique|$}{
            agregar($posibleClique$, $u$)
	 		}}

    \textbf{Entero} $maxFrontera = 0$ \\
    \textbf{Entero} $nodo = 0$ \\

    \ForAll{$v \in$ Nodos($posibleClique$)}{
	agregar($clique$, $v$) \\
		\If{frontera($clique$) $> maxFrontera$}{
		    $maxFrontera = $ frontera($clique$)\\
		    $nodo = v$
	}
	quitar($clique$, $v$) \\}


    \eIf{ $|posibleClique| =$ 0}
	{$res$ = hacerPar(0, $clique$)}
    {	agregar($clique$, $v$) \\
	$res$ = hacerPar($maxFrontera$, $clique$) }

    \textbf{devolver} $res$ \\
\end{algorithm}
Donde $frontera$ calcula la frontera del subgrafo pasado por parametro, $Nodos$ devuelve todos los nodos del subgrafo, $agregar$ inserta un elemento en un arreglo, $quitar$ quita un elemento en un arreglo, $vecindad$ nos devuelve todos los nodos adyacentes al nodo pasado por parametro y $hacerPar$ genera un par con lo dos elementos pasados por parametro. \newline

\item \textbf{Tiene todos los nodos de $S_{0}$ salvo 1 que se lo reemplaza por otro que no estaba en el:} \newline
En este subconjunto lo primero que hicimos fue quitar un nodo de $S_{0}$ y posteriormente, con la clique que nos quedo, agregarle otro. Para esto, utilizamos la funcion que quitaba un nodo a la clique y luego, utilizamos la funcion que nos calculaba el subconjunto anterior para agregar el nuevo nodo. Al hacer esto, logramos que nuestra clique no quede centrada en el nodo de mayor grado ya que este eventualmente podria dejar de formar parte de la solucion.
En el caso de que la clique tenga un solo elemento, dicidimos que este sea reemplazado por el nodo adyasente a el de mayor grado \newline
\begin{algorithm}[H]
    \SetAlgoLined
    \caption{permutarNodo}
    \KwIn{\textbf{Grafo} $grafo$, \textbf{Conj(Entero)} $clique$}
    \KwOut{\textbf{par(Entero,Conj(Entero))} $res$}
	
   \textbf{Conj(Entero)} $cliqueResultante$ = segundo(quitarNodo($grafo$, $clique$)) \\

    \eIf{$|cliqueResultante| = 0$}{
	    \textbf{Entero} $nodo$ = vecinoDeMayorGrado($grafo$,$clique[0]$) \\
            agregar($cliqueRes$, $nodo$)\\
	    $res$ = hacerPar ($|$vecindad($nodo$)$|$, $cliqueRes$)	
	 		}{$res$ = agregarNodo($grafo$, $clique$)}

    \textbf{devolver} $res$ \\
\end{algorithm}
Donde $segundo$ nos devuelve el segundo elemento de una tupla,  $agregar$ inserta un elemento en un arreglo, $vecinoDeMayorGrado$ nos da el nodo de mayor grado entre todos los vecinos del vertice pasado por parametro, $vecindad$ nos devuelve todos los nodos adyacentes al nodo pasado por parametro y $hacerPar$ genera un par con lo dos elementos pasados por parametro. \newline
\end{itemize}

Una vez que generamos a los tres candidatos de la vecindad, tomamos al que tiene mayor frontera y lo comparamos con $S_{0}$. Si la frontera de $S_{0}$ es menor, entonces la clique de frontera maxima de nuestra vecindad es nuestra nueva solución inicial. Caso contrario, el programa finaliza ya que ningún elemento de la vecindad puede mejorarla.

\subsection{Complejidad Temporal}
Para analizar la complejidad de nuestro algoritmo, vamos a separarlo en 5 funciones:
\begin{itemize}
\item \textbf{Generar grafo:} \newline
Para poder generar un grafo con los parametros de entrada, utilizamos una lista y una matriz de adyacencia para poder obtener la mayoria de las operaciones en $\mathcal{O}(1)$ y asi no aumentar la complejidad de nuestro algoritmo. Por consiguiente, la complejidad es $\mathcal{O}(n^{2})$.

\item \textbf{quitarNodo:} \newline
Como podemos observar en el pseudocodigo Nº 6, lo primero que hace nuestro algoritmo es calcular $vecindad$ del nodo 0, la cual tiene una complejidad $\mathcal{O}(1)$. Posteriormente, hay un ciclo el cual consiste en recorrer todos los nodos de la clique($\mathcal{O}(n)$) y almacenar el de menor grado en cada paso ($\mathcal{O}(1)$).
\newline
Luego, hay un ciclo el cual nuevamente recorre todos los nodos de la clique ($\mathcal{O}(n)$) y en cada paso utiliza la funcion $agregar$ la cual fue implementada con $push_back$\footnote{http://www.cplusplus.com/reference/vector/vector/push\_back/} con una complejidad de $\mathcal{O}(1)$.
\newline
Por ultimo, se calcula el valor de la frontera mediante la funcion $frontera$ la cual tiene una complejidad de $\mathcal{O}(n)$ y se genera un par con la solucion del problema. Este par es generado mediante la funcion $make\_pair$\footnote{http://www.cplusplus.com/reference/utility/make\_pair/} cuya complejidad es $\mathcal{O}(1)$.
\newline
Como se puede observar, la complejidad final de $quitarNodo$ es $\mathcal{O}(n)$

\item \textbf{agregarNodo:} \newline 
Como podemos observar en el pseudocodigo Nº 7, lo primero que hace nuestro algoritmo es crear un arreglo de n pocisiones ($\mathcal{O}(n)$)en donde va a marcar, para cada nodo, cuantos adyacentes pertenecen a la clique. Luego, recorre cada nodo de la clique y le suma uno a cada uno de sus vecinos ($\mathcal{O}(n^{2})$). Por ultimo, recorre todos los nodos de la clique y coloca su posicion en el arreglo en 0 para que no figuren como posibles candidatos ($\mathcal{O}(n)$). \newline
Una vez hecho esto, agrega en un arreglo, mediante la funcion $push_back$,\footnote{http://www.cplusplus.com/reference/vector/vector/push\_back/} todos los nodos que pueden formar una clique. Finalmente, compara la frontera ($\mathcal{O}(n)$) de todas las posibles soluciones agregando (con $push_back$\footnote{http://www.cplusplus.com/reference/vector/vector/push\_back/}) y quitando (con $pop_back$\footnote{http://www.cplusplus.com/reference/vector/vector/pop\_back/} cuya complejidad es $\mathcal{O}(1)$) los nodos que formaban una clique con la que teniamos anteriormente. el costo de realizar esto es $\mathcal{O}(n^{2})$. \newline
Una vez que tiene la solucion, genera un par con esta mediante la funcion $make\_pair$\footnote{http://www.cplusplus.com/reference/utility/make\_pair/} cuya complejidad es $\mathcal{O}(1)$.
\newline
Como se puede observar, la complejidad final de $agregarNodo$ es $\mathcal{O}(n^{2})$

\item \textbf{permutarNodo:} \newline 
Como podemos observar en el pseudocodigo Nº 8, nuestro algoritmo hace dos cosas. Primero le quita un nodo a la clique mediante la funcion $quitarNodo$ cuya complejidad es $\mathcal{O}(n)$. Posteriormente, se fija si el resultado de $quitarNodo$ es vacio o no. \newline
En el caso de que sea vacio, calcula mediante $vecinoDeMayorGrado$ cual de todos los nodos adyacentes al pasado por parametro tiene mayor grado. Para esto, recorre todos los nodos adyacentes y compara en $\mathcal{O}(1)$ sus grados, por lo tanto su complejidad es $\mathcal{O}(n)$. Una vez que tiene la solucion, la coloca en un arreglo mediante $push_back$\footnote{http://www.cplusplus.com/reference/vector/vector/push\_back/} y genera un par con esta mediante la funcion $make\_pair$\footnote{http://www.cplusplus.com/reference/utility/make\_pair/} cuya complejidad es $\mathcal{O}(1)$.
\newline
Si por el contrario, la clique no estaba vacia, obtiene la solucion mediante la funcion $agregarNodo$ cuya complejidad es $\mathcal{O}(n^{2})$
\newline
Como se puede observar, la complejidad final de $permutarNodo$ es $\mathcal{O}(n^{2})$

\item \textbf{busquedaLocal:} \newline 

\begin{algorithm}[H]
    \SetAlgoLined
    \caption{busquedaLocal}
    \KwIn{\textbf{Grafo} $grafo$, \textbf{Entero} $m$}
    \KwOut{\textbf{Conj(Entero)} $res$}
	
    agregar($res$, nodoDeMayorGrado($grafo$))\\
    	
    \For{i = 0  \textbf{to} i $<$ m}{
        \textbf{par(Entero, Conj(Entero))} $aux = $ agregarNodo($grafo, res$)\\
	\If{primero($aux$)$ < $primero(quitarNodo($grafo$, $res$))}{
            $aux = $ quitarNodo($grafo$, $res$)\\
	 		}
    	\If{primero($aux$)$ < $primero(permutarNodo($grafo$, $res$))}{
            $aux = $ permutarNodo($grafo$, $res$)\\
	 		}
	\eIf{primero($aux$)$ > $frontera($res$)}{
            $res = $ primero($aux$)\\
	 		}{$salir del ciclo$}
	}
    \textbf{devolver} $res$ \\
\end{algorithm}

LLEGUE HASTA ACA!!!!!!!!!!!!!!!!!!!!!!!!!!!

Como podemos observar en el pseudocodigo Nº 8, nuestro algoritmo hace dos cosas. Primero le quita un nodo a la clique mediante la funcion $quitarNodo$ cuya complejidad es $\mathcal{O}(n)$. Posteriormente, se fija si el resultado de $quitarNodo$ es vacio o no. \newline
En el caso de que sea vacio, calcula mediante $vecinoDeMayorGrado$ cual de todos los nodos adyacentes al pasado por parametro tiene mayor grado. Para esto, recorre todos los nodos adyacentes y compara en $\mathcal{O}(1)$ sus grados, por lo tanto su complejidad es $\mathcal{O}(n)$. Una vez que tiene la solucion, la coloca en un arreglo mediante $push_back$\footnote{http://www.cplusplus.com/reference/vector/vector/push\_back/} y genera un par con esta mediante la funcion $make\_pair$\footnote{http://www.cplusplus.com/reference/utility/make\_pair/} cuya complejidad es $\mathcal{O}(1)$.
\newline
Si por el contrario, la clique no estaba vacia, obtiene la solucion mediante la funcion $agregarNodo$ cuya complejidad es $\mathcal{O}(n^{2})$
\newline
Como se puede observar, la complejidad final de $permutarNodo$ es $\mathcal{O}(n^{2})$


\subsection{Instancias problemáticas}
\subsection{Experimentación}


\newpage
\section{Metaheurística de Búsqueda Tabú}
\subsection{Heurística de Búsqueda Tabú}

\subsubsection{Explicación del algoritmo realizado}
\subsubsection{Complejidad Temporal}
\subsubsection{Instancias problemáticas}
\subsubsection{Experimentación}


\subsection{Heurística de Búsqueda Tabú}

\begin{algorithm}[H]
    \SetAlgoLined
    \caption{TabuSearch}
    \KwIn{\textbf{Conj(nodos)} $solución\_inicial$, \textbf{Grafo} $g$, \textbf{Entero} $cantidad\_pasos$, \textbf{Entero} $desviacion\_permitida$}
    \KwOut{\textbf{Conj(nodos)} $solución\_final$}
		
	\textbf{Entero} $desviacion\_permitida\_aux$ = 0 \\
	\textbf{Conj(nodos)} $solución\_final$ = LocalSearch($solución\_inicial$, $g$)	\\		
	
	\While{$cantidad\_pasos >$ 0}{

	$solución\_inicial$ = $solución\_final$ \\
	mezclar(Candidatos\_clique($solución\_inicial$)) \\
	mezclar(Nodos($solución$\_$inicial$)) \\
	$desviacion\_actual$ = $desviacion\_permitida$ \\

	\While{ Mejore la frontera $\vee \ desviacion\_actual >$ 0}{

	 	Buscar Mejor Nodo a Agregar No Tabu () \\
		Buscar Mejor Nodo a Quitar No Tabu ( ) \\
					
	}

	Vaciar la lista Tabu \\
	Agregar los ultimos dos nodos de $solución\_final$ a la lista Tabu \\
	$cantidad\_pasos$ - 1
	}
    	\textbf{devolver} $solución\_final$ \\

\end{algorithm}

\begin{algorithm}[H]
    \SetAlgoLined
    \caption{Buscar Mejor Nodo a Agregar No Tabu}

	\ForAll{$u \in Candidatos\_clique($solución\_inicial$)$}{
	 		\If{$\neg$ es tabu($u$) $\land$ $u$ $\notin Nodos(solución\_inicial)$}{
	 			\eIf{frontera($solución\_final$) $<$ frontera( $solución\_inicial$ con $u$) }{
					$solución\_final$ = $solución\_inicial$ con $u$}{
	  				\If{$desviacion\_permitida\_aux >$ 0}{
	 				$solución\_inicial$ = $solución\_inicial$ con $u$ \\
	 				Poner en lista Tabu a $u$ \\
	 				$desviacion\_actual$ - 1}}}}

\end{algorithm}

\begin{algorithm}[H]
    \SetAlgoLined
    \caption{Buscar Mejor Nodo a Quitar No Tabu}

	\ForAll{$u \in$ Nodos($solución$\_$inicial$)}{
		\If{$\neg$ es tabu($u$)}{
	 		\eIf{frontera($solución\_final$) $<$ frontera( $solución\_inicial$ sin $u$)}{
	 			$solución\_final$ = $solución\_inicial$ sin $u$}{
				\If{$desviacion\_permitida\_aux >$ 0}{
					$solución\_inicial$ = $solución\_inicial$ sin $u$ \\
					Poner en lista Tabu a $u$ \\
					$desviacion\_actual$ - 1}}}}

\end{algorithm}

Donde:
\begin{itemize}
 \item $desviacion$\_$permitida$ Dice la cantidad de veces que se agrega o quita un nodo por iteracion (empeorando la solución parcial).
 \item $cantidad$\_$pasos$ Son la cantidad de veces que se aplica el algoritmo.
 \item frontera : Dice, dada una solución como parametro, el numero de nodos adyacentes a la frontera (es lo que pide maximizar el enunciado).
 \item Candidatos$\_$clique : Dice los nodos que pertenecen a la clique maxima de los nodos pasados como parametro.
 \item Nodos : Da los nodos pertenecientes a la solución pasada como parametro.
 \item mezclar: Mezcla de manera aleatoria un conjunto.
 \item Mejore la frontera : es esquivalente a poner $frontera\_ini <$ frontera($solución\_final$).
\end{itemize}

\begin{algorithm}[H]
    \SetAlgoLined
    \caption{LocalSearch}
    \KwIn{\textbf{Conj(nodos)} $solución\_inicial$, \textbf{Grafo} $g$ }
    \KwOut{\textbf{Conj(nodos)} $solución\_final$}

	\textbf{Conj(nodos)} $solución\_final$ = $solución\_inicial$ \\

	\While{Mejore la frontera}{
	 	\ForAll{$u \in Candidatos\_clique( $solución\_inicial$ )$}{
	 		\If{ $u$ $\notin Nodos(solución\_inicial)$ $\wedge$ frontera($solución\_final$) $<$ frontera($solución\_inicial$ con $u$)}
			{ $solución\_final$ = $solución\_inicial$ con $u$ }
		}
	
	\ForAll{ $u \in$ Nodos($solución\_inicial$) }{
		\If{$\neg$ es tabu($u$) $\wedge$ frontera($solución\_final$) $<$ frontera($solución\_inicial$ sin $u$) }
			 { $solución\_final$ = $solución\_inicial$ sin $u$ }
	}
	$solución\_inicial$ = $solución\_final$ 
	}
    	\textbf{devolver} $solución\_final$ \\

\end{algorithm}

\newpage

\section{Experimentación General}
%En el grafico se puede observar como el crecimiento de todos las heuristicas es polinomial a menida que el grafo aumenta su densidad en aristas. A su vez el algoritmo $Tabu$ $Search$ presenta una mayor diferencia en cuanto a tiempo, esto es razonable porque puede empeorar parcialmente la solucion dependiendo la cantidad de nodos, entonces esa diferencia (constante) lo que hace es "subirme" la funcion la cantidad observada. 

Para realizar la experimentación respecto a la calidad de las heurísticas presentadas, utilizamos un generador de los siguientes grafos:
\begin{itemize}
\item Estrella + CMF
\item Estrella+Puente+CMF
\item Estrella+Puente+Doble Estrella
\item Banana Tree (Palmera)
\item Rueda
\end{itemize}

 \begin{figure}[H] %[h] Aqui [b] para button [t] para top
\begin{center}
\includegraphics[width=400pt]{../imgs/calidadSolucionesChicas15.jpg}
\caption{Comparación realizada con soluciones chicas con un grafo de tipo Estrella+CMF.}
\end{center}
\end{figure}

 \begin{figure}[H] %[h] Aqui [b] para button [t] para top
\begin{center}
\includegraphics[width=400pt]{../imgs/calidadSolucionesChica14.jpg}
\caption{Comparación realizada con soluciones chicas con un grafo de tipo Estrella+Puente+CMF.}
\end{center}
\end{figure}

 \begin{figure}[H] %[h] Aqui [b] para button [t] para top
\begin{center}
\includegraphics[width=400pt]{../imgs/calidadSolucionesChicas17.jpg}
\caption{Comparación realizada con soluciones chicas con un grafo de tipo Estrella+Puente+Doble Estrella.}
\end{center}
\end{figure}

 \begin{figure}[H] %[h] Aqui [b] para button [t] para top
\begin{center}
\includegraphics[width=400pt]{../imgs/calidadSolucionesChicas3.jpg}
\caption{Comparación realizada con soluciones chicas con un grafo de tipo Banana Tree.}
\end{center}
\end{figure}

 \begin{figure}[H] %[h] Aqui [b] para button [t] para top
\begin{center}
\includegraphics[width=400pt]{../imgs/calidadSolucionesChicas2.jpg}
\caption{Comparación realizada con soluciones chicas con un grafo de tipo rueda.}
\end{center}
\end{figure}

 \begin{figure}[H] %[h] Aqui [b] para button [t] para top
\begin{center}
\includegraphics[width=400pt]{../imgs/calidadSolucionesGrandes15.jpg}
\caption{Comparación realizada con soluciones grandes con un grafo de tipo Estrella+CMF.}
\end{center}
\end{figure}

 \begin{figure}[H] %[h] Aqui [b] para button [t] para top
\begin{center}
\includegraphics[width=400pt]{../imgs/calidadSolucionesGrandes14.jpg}
\caption{Comparación realizada con soluciones grandes con un grafo de tipo Estrella+Puente+CMF.}
\end{center}
\end{figure}

 \begin{figure}[H] %[h] Aqui [b] para button [t] para top
\begin{center}
\includegraphics[width=400pt]{../imgs/calidadSolucionesGrandes3.jpg}
\caption{Comparación realizada con soluciones grandes con un grafo de tipo Banana Tree.}
\end{center}
\end{figure}

\newpage

\section{Referencias}
\begin{itemize}
\item CORMEN, THOMAS H. ; Introduction to Algorithms, Third ed. 2009. The MIT Press.
\end{itemize}


\end{document}
