\nonstopmode
\documentclass[10pt, a4paper]{article}
%\usepackage{subfig}

\parindent=20pt
\parskip=8pt
\usepackage[width=15.5cm, left=3cm, top=2.5cm, height= 24.5cm]{geometry}
\usepackage[spanish]{babel}
\usepackage[utf8]{inputenc}
\usepackage{fancyhdr}
\usepackage{multirow}
\usepackage{rotating}
\usepackage{indentfirst}
\usepackage{latexsym}
\usepackage{caratula}
\usepackage{gnuplottex}
\usepackage{epsfig}
\usepackage{lastpage}
\usepackage{amsfonts}
\usepackage{listings}
\usepackage[export]{adjustbox}
\usepackage{pdfpages}
\lstset{language=C}
\usepackage[ruled,vlined,linesnumbered]{algorithm2e}
\usepackage{graphicx}
\usepackage{float}
\usepackage{color}

\graphicspath{{imgs/}}



% Acomodo fancyhdr.
\pagestyle{fancy}
\thispagestyle{fancy}
\addtolength{\headheight}{1pt}
\lhead{Algoritmos y Estructuras de Datos III}
\rhead{TP3}
\cfoot{\thepage /\pageref{LastPage}}
\renewcommand{\footrulewidth}{0.4pt}
\renewcommand{\thesubsubsection}{\thesubsection.\alph{subsubsection}}


\author{Algoritmos y Estructuras de Datos III, DC, UBA.}
\date{}
\title{}

\begin{document}
	
\thispagestyle{empty}
\materia{Algoritmos y Estructuras de Datos III}
\submateria{Trabajo Pr\'actico N$^{\circ}$3}
\titulo{}
\integrante{Izcovich, Sabrina}{550/11}{sizcovich@gmail.com}
\integrante{Garcia Marset, Matias}{356/11}{matiasgarciamarset@gmail.com}
\integrante{Orellana, Ignacio}{229/11}{nacho@foxdev.com.ar}
\integrante{Vita, Sebastián}{149/11}{sebastian\_vita@yahoo.com.ar}

\maketitle

\tableofcontents
\newpage

\section{Introducci\'on}
Este trabajo práctico consiste en la resolución de ciertos problemas algorítmicos que cumplen con restricciones impuestas por la cátedra, como por ejemplo el orden de complejidad máximo de los mismos, entre otras. Para justificar las implementaciones de los problemas en cuestión, fue necesaria la utilización de herramientas lógico-matemáticas que serán mencionadas a lo largo del desarrollo de cada ejercicio.\newline
Para comprobar que nuestras soluciones resolvieran correctamente los problemas propuestos, debimos dividir el análisis de los mismos en secciones a fin de estudiar minuciosamente las características de éstos. Estas secciones se dividen de la siguiente forma:

\begin{itemize}
\item \textbf{Problema a resolver:} En esta sección, nos encargamos de describir detalladamente el problema a resolver dando ejemplos del mismo y sus soluciones.
\item \textbf{Resolución coloquial:} En esta parte, nos dedicamos a explicar de forma clara, sencilla, estructurada y concisa las ideas desarrolladas para la resolución del problema en cuestión. Para ello, decidimos utilizar pseudocódigo y lenguaje coloquial combinando ambas herramientas de manera adecuada.
\item \textbf{Demostración de correctitud:} Utilizamos este apartado para justificar que el punto anterior resuelve efectivamente el problema y demostramos formalmente la correctitud del mismo.
\item \textbf{Complejidad del algoritmo:} En esta sección, nos ocupamos de deducir una cota de complejidad temporal del algoritmo propuesto en función de los parámetros considerados como correctos. Por otro lado, justificamos por qué el algoritmo desarrollado para la resolución del problema cumple con la cota dada.
\item \textbf{Instancias posibles:} Este sector presenta un conjunto de instancias que permiten verificar la correctitud del programa implementado cubriendo todos los casos posibles y justificando la elección de los mismos. Dichas instancias fueron evaluadas por el algoritmo realizado y los resultados obtenidos fueron comprobados.
\item \textbf{Experimentación:} Por último, los tests consistieron en experimentaciones computacionales utilizadas para medir la performance del programa implementado. Para ello, debimos preparar un conjunto de casos de test que permitieran observar los tiempos de ejecución en función de los parámetros de entrada que fueran relevantes. Para ello, nos encargamos de generar instancias aleatorias como también particulares. Para que los resultados fueran visibles y claros, utilizamos una comparación gráfica entre los tiempos medidos y la complejidad teórica calculada.
\item \textbf{Código fuente:} En este apéndice, presentamos las funciones relevantes del código fuente que implementa la solución propuesta. Para ello, decidimos utilizar el lenguaje \verb*#C++# dado que éste cuenta con la librería $stl$ que proporciona las estructuras necesarias para la realización de dicha tarea.
\end{itemize}
\newpage

\section{Ejercicio 1}
En este ejercicio, nos limitamos a pensar situaciones de la vida real que pudieran ser modeladas con $cliques\ de\ máxima\ frontera$.
\begin{itemize}
\item Juanita es una apasionada de conocer calles nuevas y desea mudarse a algún barrio de una ciudad que le permita seguir con su pasión. Pero Juanita trabaja en el campo, o sea que fuera de cualquier ciudad, así que se le ocurre seleccionar su próxima vivienda de acuerdo a la cantidad de calles que va a poder conocer para llegar al campo. Podemos modelar el problema como los nodos representantes de las esquinas y las aristas las interconexiones entre éstas. Llamamos ciudad a todas las esquinas que se conectan todas con todas, luego, a una clique y campo a todo lo que se encuentra entre una ciudad y otra. Por lo tanto, se desea encontrar una ciudad con mayor frontera para que la cantidad de posibles calles nuevas a recorrer fuera de la ciudad sea la mayor.

\item Se vienen las elecciones y como todos los años, el futuro candidato a presidente Josecito quiere invertir en repavimentación. Esta vez quiere elegir las esquinas interconectadas entre sí que tengan la mayor cantidad de calles en su frontera. Esto se debe a que las máquinas de repavimentación, que se encuentran en cada esquina de la zona de trabajo, necesitan estar conectadas entre sí dado que, cuando se quedan sin algún material necesitan llegar rápidamente a otra de las máquinas para terminar velozmente el trabajo. Si modelamos a las esquinas y a las aristas como calles, Josecito debe encontrar la clique de mayor frontera. 
\end{itemize}

\newpage

\section{Ejercicio 2}
\subsection{Algoritmo Implementado}

Para la resolución exacta del problema a resolver, realizamos un algoritmo que recorriera todas las posibles cliques del grafo de entrada. Esto se debió a que no hallamos ninguna caracterización que nos permitiera reducir el espacio de búsqueda de cliques para que el algoritmo ignorara una cantidad asintótica de ellas. Luego, nuestra implementación genera todas las soluciones siguiendo un orden lexicográfico para evitar repetir ejecuciones. Por ejemplo, la clique $k_{2}$ formada por los nodos \{1,2\} es la misma que se forma al elegir los nodos \{2,1\}. De esta forma, al representar los nodos como números y formar cliques respetando su orden lexicográfico, nos ahorramos pasar por soluciones que ya formamos antes. %para mi hay que decir que chequea si ya lo ejecuto, porque el orden lexicografico solo me dice cual va a verificar primero...

Luego, nuestro algoritmo comienza generando todas las cliques que contienen al nodo de menor valor numérico que no haya sido utilizado y, en cada paso, va a seleccionar una serie de candidatos a extender la clique actual hasta no encontrar más y volver hacia atras, quitando el último nodo que se agregó a la clique, y eligiendo el próximo nodo que sea candidato si es que lo hubiera. El orden lexicografico es garantizado al momento de elegir los candidatos, ya que solo elegira aquellos candidatos cuyo numero de nodo sea mayor al numero de nodo del ultimo nodo que fue agregado a la clique.

A continuación, se expone el pseudocódigo del algoritmo:

Variables globales: \textbf{Conj(nodo)} $cliqueMaxima$, \textbf{Entero} $fronteraMaxima$.\\

\begin{algorithm}[H]
	\SetAlgoLined
	\caption{Exacto}
	\KwIn{\textbf{Grafo} $g$} %esto hay que ponerlo bien
	\KwOut{\textbf{Conj(nodo)} $clique$}

	\textbf{Conj(nodo)} $clique = \emptyset$\\
	\textbf{Conj(nodo)} $candidatos = $ nodos($g$)\\

	Backtracking($clique$, $candidatos$)\\

	\textbf{devuelvo} $cliqueMaxima$\\

\end{algorithm}

\begin{algorithm}[H]
	\SetAlgoLined
	\caption{Backtracking}
	\KwIn{\textbf{Conj(nodo)} $clique$, \textbf{Conj(nodo)} $candidatos$, \textbf{Grafo} $g$} %esto hay que ponerlo bien
	\KwOut{\textbf{Conj(nodo)} $clique$}

	\For{$c \in candidatos$}{
		\If{\textbf{frontera}($ clique \cup \{c\} $) es mayor a $fronteraMaxima$ }{
			$cliqueMaxima = clique \cup \{c\} $\\
			$fronteraMaxima = $frontera($clique \cup \{c\}$)\\
		}
		\If{\textbf{frontera}($ clique \cup \{c\} $) $>$ \textbf{frontera}($clique$)}{
			Backtracking($ clique \cup \{c\} $, calcularCandidatos($ clique \cup \{c\} $), $g$, $c$)\\
		}
	}
\end{algorithm}

\begin{algorithm}[H]
	\SetAlgoLined
	\caption{frontera}
	\KwIn{\textbf{Conj(nodo)} $clique$, \textbf{Grafo} $g$} %esto hay que ponerlo bien
	\KwOut{\textbf{Entero} $res$}

	$res = 0$\\
	\For{\textbf{nodo} $v \in nodos(g)$}{
		$res = res + (\#adyacentes(v) - (\#clique - 1))$\\
	}

	\textbf{devuelvo} $res$\\

\end{algorithm}

\begin{algorithm}[H]
	\SetAlgoLined
	\caption{calcularCandidatos}
	\KwIn{\textbf{Conj(nodo)} $clique$, \textbf{Grafo} $g$, \textbf{Nodo} $ultimoNodoAgregado$} %esto hay que ponerlo bien
	\KwOut{\textbf{Conj(nodo)} $candidatos$}

	nodo $w$ = un elemento de $clique$\\

	$candidatos = adyacentes(w)$\\
	\For{\textbf{todo nodo} $v \in nodos(g), v \neq w$}{
		$candidatos$ = interseccion entre $adyacentes(v)$ y $candidatos$\\
	}

	quitar de $candidatos$ todos los nodos menores numericamente a $ultimoNodoAgregado$

	\textbf{devuelvo} $candidatos$\\

\end{algorithm}

En este caso el programa comienza con la funcion Exacto, esta es la encargada de crear los valores inciales con los cuales se va a llamara a la funcion backtracking, luego esta hace las llamdas recursivas para ir creando las diferentes cliques y sus candidatos teniendo en cuenta el orden lexicografico y en cada paso chequeando si las cliques que considera son mejores que la mejor encontrada actualmente. \\
La funcion $calcularCandidatos$ busca aquellos nodos que no son los de la clique actual pero que estan conectados con todos ellos, eso se logra calculando la interseccion entre los nodos adyacentes a los nodos de la clique. \\
La funcion $frontera$ devuelve el valor numerico de la frontera de una clique, como tenemos por pre-condicion que todo conjunto de nodos que reciba como parametro la funcion es una clique, entonces calcular la frontera se puede hacer restandole a la cantidad de nodos adyacentes de cada nodo la cantidad de nodos de la clique menos 1. 

\subsection{Análisis de complejidad}

El analisis de complejidad del algoritmo posee una cota superior definida por la funcion recursiva de backtracking, esta funcion se pueden analizar calcular el arbol de soluciones del problema en cuestion. Como buscamos cliques y puede haber cliques de tamaño 1 hasta tamaño $n$ inclusive, entonces el arbol de soluciones va a tener un tamaño de 
\newpage

\section{Ejercicio 3}
\subsection{Heurística Constructiva Golosa}
Para resolver el algoritmo presentado anteriormente con una técnica golosa, decidimos utilizar el procedimiento que se presenta a continuación:\newline
\newline
\begin{algorithm}[H]
    \SetAlgoLined
    \caption{HeurísticaGolosa}
    \KwIn{\textbf{Entero} $cantVertices$, \textbf{Entero} $cantAristas$, \textbf{Grafo} $g$} %esto hay que ponerlo bien
    \KwOut{\textbf{Conj(nodos)} $clique$}
	Entero $nodoDeMayorGrado$ = nodoDeMasGrado($g$)\\
	Grafo $cliqueHastaAhora$ = $\emptyset$\\
	agregar($v$, $cliqueHastaAhora$)\\
	\ForAll{$u \in$ adyacentes(nodoDeMayorGrado, nodos($g$))}{
		\If{forma una clique($g$, agregar($u$, $cliqueHastaAhora$)) $\land$ frontera($g$, agregar($u$, $cliqueHastaAhora$) $>$ frontera($cliqueHastaAhora$)}{
		agregar($u$, $cliqueHastaAhora$)}}
\textbf{devolver} $cliqueHastaAhora$
\end{algorithm}

donde $nodoDeMasGrado$ consiste en una función que toma el nodo del grafo cuyo grado es el mayor, $frontera$ es una función que calcula la frontera de un conjunto de nodos dentro de un grafo y $adyacentes$ consiste en una lista de nodos adyacentes a un determinado nodo.\newline


Hay que decir que es asíntico, que se rompe muy fácil, como por ejemplo cuando la clique de maxima frontera no contiene al nodo de mayor grado.

\subsection{Heurística de Búsqueda Local}

Los algoritmos de búsqueda local parten de una solución inicial $S_{0}$ y en cada paso intentan mejorarla. Para esto, se calculan todas las posibles variaciones de $S_{0}$ que forman una solución al problema. Al conjunto de todas estas se lo llama vecindad. 
Estos algoritmos se ejecutan siempre y cuando exista una solución, perteneciente a la vecindad, que sea mejor a la que ya teníamos. \newline \newline
Para nuestro ejercicio, utilizamos como $S_{0}$ al nodo 1. Dicha elección se debió a que un nodo forma una clique, con lo cual es una posible solución a nuestro problema. Además, sabemos que este existe para cualquier grafo ya que la menor cantidad de nodos que se pueden ingresar en nuestro programa es uno. 
\newline Por otro lado, decidimos separar la vecindad en 3 subconjuntos y para cada uno de estos, tomar la clique de mayor frontera. Al hacer esto, nos evitamos tener que almacenar todas las posibles cliques para posteriormente elegir la mejor. Los tres subconjuntos se formaron con las cliques que cumplen lo siguiente:
\begin{itemize}
\item \textbf{Tiene todos los nodos de $S_{0}$ salvo 1:} \newline En lugar de calcular todas las posibles cliques de este subconjunto, decidimos quedarnos únicamente con la clique que se obtiene de quitarle el nodo de menor grado a $S_{0}$. Esto se debe a que la frontera de una clique se puede calcular con la siguiente formula:\newline
Sea $S_{0}$ = ($V$,$E$) y n = $|$$V$$|$
\begin{equation}
  \delta(S_{0}) = \sum_{v \in V}^{} d(v) - n*(n-1)
\end{equation}
Ahora, si quitamos un nodo $v'$ de $S_{0}$, obtenemos lo siguiente:
\begin{equation}
  \delta(S'_{0}) = \sum_{v \in V/v'}^{} d(v) - (n-1)*(n-2)
\end{equation}
Como nosotros queremos encontrar el mayor \delta$(S'_{0})$, debemos quedarnos con el que tiene la sumatoria de mayor valor ya que (n-1)*(n-2) es igual para todas las cliques. Con lo cual, el nodo que debemos eliminar para poder maximizar la sumatoria, es el de menor grado.
\item \textbf{Tiene todos los nodos de $S_{0}$ mas uno que no pertenecía a $S_{0}$:} \newline
Para poder obtener la clique de frontera máxima de este subconjunto, buscamos todos los nodos del grafo que pueden formar una clique con $S_{0}$. Para esto, contamos cuantos nodos de la clique se conectan a cada nodo del grafo. Si un nodo es alcanzado por todos los que pertenecen a $S_{0}$, entonces ese nodo forma una clique. Una vez que tenemos a todos los posibles candidatos, calculamos la frontera de cada uno y nos quedamos con la mayor.
\item \textbf{Tiene todos los nodos de $S_{0}$ salvo 1 que se lo reemplaza por otro que no estaba en $S_{0}$:} \newline
\end{itemize}
Una vez que generamos la vecindad, tomamos la clique de mayor frontera, $S_{maxima}$ y la comparamos con $S_{0}$. Si la frontera de $S_{0}$ es menor, $S_{maxima}$ es nuestra nueva solución inicial. Caso contrario, el programa finaliza ya que ningún elemento de la vecindad puede mejorarla.



%no entendi esto! 
%Hay que ver cuánto dura cada vértice en cantidad de ejecuciones.


\subsection{Heurística de Búsqueda Tabú}

\begin{algorithm}[H]
    \SetAlgoLined
    \caption{TabuSearch}
    \KwIn{\textbf{Conj(nodos)} $solución\_inicial$, \textbf{Grafo} $g$, \textbf{Entero} $cantidad\_pasos$, \textbf{Entero} $desviacion\_permitida$}
    \KwOut{\textbf{Conj(nodos)} $solución\_final$}
		
	\textbf{Entero} $desviacion\_permitida\_aux$ = 0 \\
	\textbf{Conj(nodos)} $solución\_final$ = $solución\_inicial$ \\
		
	\While{$cantidad\_pasos >$ 0}{
		\textbf{Entero} $frontera\_ini$ = frontera($solución\_inicial$) \\
	 	\For{$u \in Candidatos\_clique(g,u)$}{
	 		\If{$\neg$ es tabu($u$) $\land \neg$ esta agregado $u$ en $solución\_inicial$}{
	 			\eIf{frontera($solución\_final$) $<$ Frontera con $u$ en $solución\_inicial$}{
					$solución\_final$ = $solución\_inicial$ con $u$}{
	  				\If{$desviacion\_permitida\_aux >$ 0}{
	 				$solución\_inicial$ = $solución\_inicial$ con $u$ \\
	 				Poner en lista Tabu a $u$ \\
	 				$desviacion\_permitida\_aux$ - 1}}}}}
	
	\ForAll{$u \in$ Nodos($solución$\_$inicial$)}{
		\If{$\neg$ es tabu($u$)}{
	 		\eIf{frontera($solución\_final$) $<$ Frontera sin $u$ en $solución\_inicial$}{
	 			$solución\_final$ = $solución\_inicial$ sin $u$}{
				\If{$desviacion\_permitida\_aux >$ 0}{
					$solución\_inicial$ = $solución\_inicial$ sin $u$ \\
					Poner en lista Tabu a $u$ \\
					$desviacion\_permitida\_aux$ - 1}}}}
							
	\If{$desviacion\_permitida\_aux \leq$ 0}{
		$solución\_inicial$ = $solución\_final$}
	\While{$frontera\_ini <$ frontera($solución\_final$) $\vee \ desviacion\_permitida\_aux >$ 0}{
		$desviacion\_permitida\_aux$ = $desviacion\_permitida$ \\
		Vaciar la lista Tabu \\
		Agregar los ultimos dos nodos de $solución\_final$ a la lista Tabu \\
		$cantidad\_pasos$ - 1}
    	\textbf{devolver} $solución\_final$ \\

\end{algorithm}

Donde:
\begin{itemize}
 \item $desviacion$\_$permitida$ dice la cantidad de veces que se agrega o quita un nodo por iteracion (empeorando la solución parcial).
 \item $cantidad$\_$pasos$ son la cantidad de veces que se aplica el algoritmo. Tener en cuenta que la primera iteracion $desviacion$\_$permitida$ es 0, por lo que se toma el maximo local.
 \item frontera : Dice, dada una solución como parametro, el numero de nodos adyacentes a la frontera (es lo que pide maximizar el enunciado).
 \item Candidatos$\_$clique : Dice los nodos que pertenecen a la clique del nodo pasado como parametro.
 \item Nodos : da los nodos pertenecientes a la solución pasada como parametro.
\end{itemize}

\newpage

\section{Ejercicio 4}
%En el grafico se puede observar como el crecimiento de todos las heuristicas es polinomial a menida que el grafo aumenta su densidad en aristas. A su vez el algoritmo $Tabu$ $Search$ presenta una mayor diferencia en cuanto a tiempo, esto es razonable porque puede empeorar parcialmente la solucion dependiendo la cantidad de nodos, entonces esa diferencia (constante) lo que hace es "subirme" la funcion la cantidad observada. 

Para realizar la experimentación respecto a la calidad de las heurísticas presentadas, utilizamos un generador de los siguientes grafos:
\begin{itemize}
\item Estrella + CMF
\item Estrella+Puente+CMF
\item Estrella+Puente+Doble Estrella
\item Banana Tree (Palmera)
\item Rueda
\end{itemize}

 \begin{figure}[H] %[h] Aqui [b] para button [t] para top
\begin{center}
\includegraphics[width=400pt]{../imgs/calidadSolucionesChicas15.jpg}
\caption{Comparación realizada con soluciones chicas con un grafo de tipo Estrella+CMF.}
\end{center}
\end{figure}

 \begin{figure}[H] %[h] Aqui [b] para button [t] para top
\begin{center}
\includegraphics[width=400pt]{../imgs/calidadSolucionesChica14.jpg}
\caption{Comparación realizada con soluciones chicas con un grafo de tipo Estrella+Puente+CMF.}
\end{center}
\end{figure}

 \begin{figure}[H] %[h] Aqui [b] para button [t] para top
\begin{center}
\includegraphics[width=400pt]{../imgs/calidadSolucionesChicas17.jpg}
\caption{Comparación realizada con soluciones chicas con un grafo de tipo Estrella+Puente+Doble Estrella.}
\end{center}
\end{figure}

 \begin{figure}[H] %[h] Aqui [b] para button [t] para top
\begin{center}
\includegraphics[width=400pt]{../imgs/calidadSolucionesChicas3.jpg}
\caption{Comparación realizada con soluciones chicas con un grafo de tipo Banana Tree.}
\end{center}
\end{figure}

 \begin{figure}[H] %[h] Aqui [b] para button [t] para top
\begin{center}
\includegraphics[width=400pt]{../imgs/calidadSolucionesChicas2.jpg}
\caption{Comparación realizada con soluciones chicas con un grafo de tipo rueda.}
\end{center}
\end{figure}

 \begin{figure}[H] %[h] Aqui [b] para button [t] para top
\begin{center}
\includegraphics[width=400pt]{../imgs/calidadSolucionesGrandes15.jpg}
\caption{Comparación realizada con soluciones grandes con un grafo de tipo Estrella+CMF.}
\end{center}
\end{figure}

 \begin{figure}[H] %[h] Aqui [b] para button [t] para top
\begin{center}
\includegraphics[width=400pt]{../imgs/calidadSolucionesGrandes14.jpg}
\caption{Comparación realizada con soluciones grandes con un grafo de tipo Estrella+Puente+CMF.}
\end{center}
\end{figure}

 \begin{figure}[H] %[h] Aqui [b] para button [t] para top
\begin{center}
\includegraphics[width=400pt]{../imgs/calidadSolucionesGrandes3.jpg}
\caption{Comparación realizada con soluciones grandes con un grafo de tipo Banana Tree.}
\end{center}
\end{figure}

\newpage

\section{Referencias}
\begin{itemize}
\item CORMEN, THOMAS H. ; Introduction to Algorithms, Third ed. 2009. The MIT Press.
\end{itemize}


\section{Código fuente}


\end{document}
