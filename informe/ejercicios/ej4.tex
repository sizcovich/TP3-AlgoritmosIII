


\begin{algorithm}[H]
    \SetAlgoLined
    \caption{TabuSearch}
    \KwIn{Conj(nodos) $solucion$\_$inicial$, Grafo $g$, Entero $cantidad$\_$pasos$, Entero $desviacion$\_$permitida$}
    \KwOut{Conj(nodos) $solucion$\_$final$}
		
		\textbf{Entero} desviacion$\_$permitida_aux = 0 \\
		
		\textbf{Mientras} ($cantidad$\_$pasos$ $>$ 0) \\
		     \textbf{Hacer} \\
							\textbf{Entero} $frontera$\_$ini$ = frontera($solucion$\_$inicial$) \\
							
							\textbf{Para} ( u \in Candidatos$\_$clique(g,u) ) \\
							       \textbf{Si} ( \neg es tabu(u) \wedge \neg esta agregado $u$ en $solucion$\_$inicial$ ) \\
				                         \textbf{Si} (Frontera agregando $u$ $\>$ Frontera sin $u$ en $solucion$\_$inicial$) \\
																              $solucion$\_$final$ = $solucion$\_$inicial$ con $u$ \\
																							Poner en lista Tabu a $u$ \\
																 \textbf{Sino} \\
																              \textbf{Si} (desviacion$\_$permitida$\_$aux > 0) \\
																							          $solucion$\_$inicial$ = $solucion$\_$inicial$ con $u$ \\
																												Poner en lista Tabu a $u$ \\
																 \textbf{Fin Si} \\
										 \textbf{Fin Si} \\
							\textbf{Fin Para} \\
							\textbf{Para} ( u \in Nodos($solucion$\_$inicial$) ) \\
							       \textbf{Si} ( \neg es tabu(u) ) \\
				                         \textbf{Si }(Frontera agregando $u$ $>$ Frontera sin $u$ en $solucion$\_$inicial$) \\
																              $solucion$\_$final$ = $solucion$\_$inicial$ sin $u$ \\
																							Poner en lista Tabu a $u$ \\
																 \textbf{Sino} \\
																              \textbf{Si} ($desviacion$\_$permitida$\_$aux$ > 0) \\
																							          $solucion$\_$inicial$ = $solucion$\_$inicial$ sin $u$ \\
																												Poner en lista Tabu a $u$ \\
																												$desviacion$\_$permitida$\_$aux$  - 1 \\
																 \textbf{Fin Si} \\
										 \textbf{Fin Si} \\
							\textbf{Fin Para} \\
							
							\textbf{Si} ($desviacion$\_$permitida$\_$aux$ \leq 0) \\
							      $solucion$\_$inicial$ = $solucion$\_$final$ \\
				    
				\textbf{Mientras}( $frontera$\_$ini$ $<$ frontera($solucion$\_$final$)  || $desviacion$\_$permitida$\_$aux$ > 0 ) \\
				
				$desviacion$\_$permitida$\_$aux$ = $desviacion$\_$permitida$  \\
				Vaciamos la lista Tabu \\
				Agregamos los ultimos dos nodos de $solucion$\_$final$ a la lista Tabu \\
				$cantidad$\_$pasos$ - 1 \\
	 \textbf{Fin Mientras} \\
    \textbf{devolver} $solucion$\_$final$ \\
		
\end{algorithm}

Donde:
\begin{itemize}
 \item $desviacion$\_$permitida$ dice la cantidad de veces que se agrega o quita un nodo por iteracion (empeorando la solucion parcial).
 \item $cantidad$\_$pasos$ son la cantidad de veces que se aplica el algoritmo. Tener en cuenta que la primera iteracion $desviacion$\_$permitida$ es 0, por lo que se toma el maximo local.
 \item frontera : Dice, dada una solucion como parametro, el numero de nodos adyacentes a la frontera (es lo que pide maximizar el enunciado).
 \item Candidatos$\_$clique : Dice los nodos que pertenecen a la clique del nodo pasado como parametro.
 \item Nodos : da los nodos pertenecientes a la solucion pasada como parametro.
\end{itemize}