%En el grafico se puede observar como el crecimiento de todos las heuristicas es polinomial a menida que el grafo aumenta su densidad en aristas. A su vez el algoritmo $Tabu$ $Search$ presenta una mayor diferencia en cuanto a tiempo, esto es razonable porque puede empeorar parcialmente la solucion dependiendo la cantidad de nodos, entonces esa diferencia (constante) lo que hace es "subirme" la funcion la cantidad observada. 

Para realizar la experimentación respecto a la calidad de las heurísticas presentadas, utilizamos un generador de las siguientes familias de grafos:
\begin{itemize}
\item Aleatorio
 \begin{figure}[H] %[h] Aqui [b] para button [t] para top
\begin{center}
\includegraphics[width=250pt]{../imgs/aleatorio.png}
\caption{Grafo de ejemplo de tipo Aleatorio.}
\end{center}
\end{figure}
 Para este tipo de grafos hicimos un algoritmo que, dada una cantidad de nodos y una densidad (entre 0 y 1), agarrara nodos de manera aleatoria y los hiciera adyacentes hasta cumplir con la cantidad de aristas buscada. Ésta consiste en el cuadrado de los nodos por la densidad.

\item Estrella + CMF
 \begin{figure}[H] %[h] Aqui [b] para button [t] para top
\begin{center}
\includegraphics[width=200pt]{../imgs/Estrella+CMF.jpg}
\caption{Grafo de ejemplo de tipo Estrella con una CMF que no esta en la estrella.}
\end{center}
\end{figure}
\item Estrella+Puente+CMF
 \begin{figure}[H] %[h] Aqui [b] para button [t] para top
\begin{center}
\includegraphics[width=200pt]{../imgs/Estrella+Puente+CMF.jpg}
\caption{Grafo de ejemplo de tipo Estrella con un puente a otra parte del grafo que tiene la CMF}
\end{center}
\end{figure}
\item Estrella+Puente+Doble Estrella
 \begin{figure}[H] %[h] Aqui [b] para button [t] para top
\begin{center}
\includegraphics[width=200pt]{../imgs/Estrella+puente+dobleEstrella.jpg}
\caption{Grafo de ejemplo de tipo Estrella con un puente a una estrella doble.}
\end{center}
\end{figure}

\item Banana Tree (Palmera)
 \begin{figure}[H] %[h] Aqui [b] para button [t] para top
\begin{center}
\includegraphics[width=200pt]{../imgs/banana.jpg}
\caption{Grafo de ejemplo de tipo Estrella con un puente a una estrella doble.}
\end{center}
\end{figure}
\item Rueda
 \begin{figure}[H] %[h] Aqui [b] para button [t] para top
\begin{center}
\includegraphics[width=150pt]{../imgs/rueda.jpg}
\caption{Grafo de ejemplo de tipo Estrella con un puente a una estrella doble.}
\end{center}
\end{figure}


\end{itemize}
Para la realización de los grafos anteriores, utilizamos un generador que los crea respetando sus propiedades. Éstas consisten en la densidad de las aristas y la disposición de las mismas. Luego, para la experimentación procedimos aumentando la cantidad de nodos de los grafos, que a su vez aumenta la cantidad de aristas de los mismos al encontrarse éstos caracterizados por un porcentaje determinado de aristas con respecto a sus nodos. De este modo, realizamos pruebas para cada heurística sobre cada familia de grafos.

Por otro lado, separamos las pruebas de calidad de soluciones en chicas y grandes, esto lo hicimos ya que nos interesa ver la solucion obtenida por el algoritmo exacto para poder compararlo con las heurísticas, y en otro contexto tambien comparar la calidad de las soluciones de las heurísticas para grafos donde el algoritmo exacto no puede resolver a tiempo. 

Las pruebas chicas consisten en grafos de entre 10 y 150 nodos con una densidad de aristas del 50\%. De esta forma, el algoritmo exacto puede encontrar soluciones en un tiempo relativamente rápido y nos permite compararlo con las heurísticas. En el caso de las pruebas grandes, tomamos grafos de entre 200 y 2000 nodos con una densidad de aristas del 50\%. La decisión acerca de la cantidad de aristas se realizó de forma tal a encontrar un balance entre grafos con pocas y muchas aristas, sin afectar demasiado el tiempo de ejecución del algoritmo exacto y de las heurísticas. 


 \begin{figure}[H] %[h] Aqui [b] para button [t] para top
\begin{center}
\includegraphics[width=500pt]{../imgs/calidadSolucionesChicas15.jpg}
\caption{Comparación realizada con soluciones chicas con un grafo de tipo Estrella+CMF.}
\end{center}
\end{figure}

En este caso, tenemos una estrella cuyo nodo central no forma parte de la clique de frontera máxima, con lo cual aquellos algoritmos que consideren a los nodos de mayor grado para empezar su ejecución y llegar a un resultado se verán perjudicados por esta familia. Tanto la heurística de búsqueda local como la metaheurística Tabú se ven perjudicadas por la caracterización del grafo, esto se debe a que la heurística parte del nodo de mayor grado y se mueve por una vecindad pequeña y nunca llega a encontrar la CMF que se encuentra a una distancia considerable de la estrella que forma el nodo. En el caso de la metaheurística, lo que sucede es que parte de una solución incial dada por la búsqueda local, con lo cual tiene un punto de partida poco ventajoso que termina provocando que genere un mal resultado. En el caso de la búsqueda golosa, ésta se ve beneficiada ya que si bien empieza con el nodo de mayor grado, eventualmente prueba con otros nodos teniendo así mayor probabilidad de encontrar la clique de frontera máxima. 

 \begin{figure}[H] %[h] Aqui [b] para button [t] para top
\begin{center}
\includegraphics[width=500pt]{../imgs/calidadSolucionesChica14.jpg}
\caption{Comparación realizada con soluciones chicas con un grafo de tipo Estrella+Puente+CMF.}
\end{center}
\end{figure}

En este caso sucede lo mismo que en el anterior, los algoritmos inician en la estrella ya que ésta tiene mayor grado y provoca que nunca se logren mover hasta las soluciones que se encuentran del otro lado del puente que se forma entre la estrella y la estructura que contiene una clique de frontera máxima del otro lado.

 \begin{figure}[H] %[h] Aqui [b] para button [t] para top
\begin{center}
\includegraphics[width=500pt]{../imgs/calidadSolucionesChicas17.jpg}
\caption{Comparación realizada con soluciones chicas con un grafo de tipo Estrella+Puente+Doble Estrella.}
\end{center}
\end{figure}


En este caso, el grafo se caracteriza por contener una estrella que está conectada con otras dos. Éstas dos estrellas forman una clique de frontera máxima que no contiene al nodo de la primera estrella. Lo que sucede es que la frontera de la estrella se aproxima a la solución correcta del problema pero no llega a serla, por eso las heurísticas se quedan con ella mientras que el exacto encuentra la óptima, que es la clique entre las estrellas que están unidas.


 \begin{figure}[H] %[h] Aqui [b] para button [t] para top
\begin{center}
\includegraphics[width=500pt]{../imgs/calidadSolucionesChicas3.jpg}
\caption{Comparación realizada con soluciones chicas con un grafo de tipo Banana Tree.}
\end{center}
\end{figure}

En este caso, tenemos un grafo de tipo banana tree, las heurísticas en la mayoría de los casos obtuvieron soluciones correctas.

 \begin{figure}[H] %[h] Aqui [b] para button [t] para top
\begin{center}
\includegraphics[width=500pt]{../imgs/calidadSolucionesChicas2.jpg}
\caption{Comparación realizada con soluciones chicas con un grafo de tipo rueda.}
\end{center}
\end{figure}

En este caso, todos los nodos del grafo forman un circuito simple salvo uno que se conecta con todos. Como la solución exacta está conformada por el nodo central y uno de sus adyacentes, todos los algoritmos llegan a la solución correcta sin mayor dificultad dado que el nodo central es el de mayor grado.

 \begin{figure}[H] %[h] Aqui [b] para button [t] para top
\begin{center}
\includegraphics[width=500pt]{../imgs/calidadSolucionesGrandes15.jpg}
\caption{Comparación realizada con soluciones grandes con un grafo de tipo Estrella+CMF.}
\end{center}
\end{figure}

En este caso, tenemos una estrella cuyo nodo central se encuentra a una distancia considerable de la clique de frontera máxima. Por lo tanto, la heurística de búsqueda local se ve perjudicada por la caracterización del grafo, esto se debe a que parte del nodo de mayor grado y se mueve por una vecindad pequeña, con lo cual nunca llega a encontrar la CMF. En el caso de la metaheurística Tabú, parte de una solución inicial dada por la búsqueda local, con lo cual tiene un punto de partida poco ventajoso que termina provocando que genere un mal resultado para algunos casos. En el caso de la búsqueda golosa, ésta se ve beneficiada ya que si bien empieza con el nodo de mayor grado, eventualmente prueba con otros nodos teniendo así mayor probabilidad de encontrar la clique de frontera máxima. 

 \begin{figure}[H] %[h] Aqui [b] para button [t] para top
\begin{center}
\includegraphics[width=500pt]{../imgs/calidadSolucionesGrandes14.jpg}
\caption{Comparación realizada con soluciones grandes con un grafo de tipo Estrella+Puente+CMF.}
\end{center}
\end{figure}

En este gráfico, se puede observar que para casos de grafos grandes de tipo Estrella+Puente+CMF, los algoritmos de Búsqueda Golosa y Local obtienen resultados muy similares. Dichos resultados consisten en un valor de frontera muy inferior al obtenido con la Heurística Tabú en gran parte de los casos. Esto se debe a que en gran parte de los grafos de este tipo la clique de máxima frontera no contiene al nodo de mayor grado y éste se encuentra alejado de la solución (que se encuentra del otro lado del puente). Luego, los valores que se obtienen con las heurísticas que empiezan con dicho nodo son incorrectos.

 \begin{figure}[H] %[h] Aqui [b] para button [t] para top
\begin{center}
\includegraphics[width=500pt]{../imgs/calidadSolucionesGrandes3.jpg}
\caption{Comparación realizada con soluciones grandes con un grafo de tipo Banana Tree.}
\end{center}
\end{figure}

En este caso, podemos ver que las tres heurísticas encuentran resultados similares. Esto se debe a que la clique de máxima frontera del grafo de tipo Banana Tree es el nodo que se encuentra entre los dos caminos simples que lo conforman, luego el de mayor grado. Por lo tanto las heurísticas encuentran la solución exacta en la mayoría de los casos.

