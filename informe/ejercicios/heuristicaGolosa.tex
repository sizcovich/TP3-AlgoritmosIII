\subsection{Explicación del algoritmo realizado}
Para resolver el algoritmo presentado anteriormente con una técnica golosa, decidimos utilizar el procedimiento que se presenta a continuación:\newline
\newline
\begin{algorithm}[H]
    \SetAlgoLined
    \caption{HeurísticaGolosa}
    \KwIn{\textbf{Entero} $cantVertices$, \textbf{Entero} $cantAristas$, \textbf{Grafo} $g$} %esto hay que ponerlo bien
    \KwOut{\textbf{Conj(nodos)} $clique$}
	Entero $nodoDeMayorGrado$ = nodoDeMasGrado($g$)\\
	Grafo $cliqueHastaAhora$ = $\emptyset$\\
	agregar($v$, $cliqueHastaAhora$)\\
	\ForAll{$u \in$ adyacentes(nodoDeMayorGrado, nodos($g$))}{
		\If{forma una clique($g$, agregar($u$, $cliqueHastaAhora$)) $\land$ frontera($g$, agregar($u$, $cliqueHastaAhora$) $>$ frontera($cliqueHastaAhora$)}{
		agregar($u$, $cliqueHastaAhora$)}}
\textbf{devolver} $cliqueHastaAhora$
\end{algorithm}

donde $nodoDeMasGrado$ consiste en una función que toma el nodo del grafo cuyo grado es el mayor, $frontera$ es una función que calcula la frontera de un conjunto de nodos dentro de un grafo y $adyacentes$ consiste en una lista de nodos adyacentes a un determinado nodo.\newline


Hay que decir que es asíntico, que se rompe muy fácil, como por ejemplo cuando la clique de maxima frontera no contiene al nodo de mayor grado.
\subsection{Complejidad Temporal}
\subsection{Instancias problemáticas}
\subsection{Experimentación}



