\subsection{Heurística Constructiva Golosa}
Idea:
\begin{itemize}
\item Agarro nodo $v$ de mayor grado
\item Meto $v$ en la bolsa de nodos seleccionados
\item Agarro el nodo de mayor grado adyacente a $v$ que forme una clique con todos los de la bolsa y que tenga mayor frontera
\item Meto en la bolsa el elegido
\item vuelvo al segundo 'Agarro' hasta que ninguno cumpla la clique
\end{itemize}


Hay que decir que es asíntico, que se rompe muy fácil, como por ejemplo cuando la clique de maxima frontera no contiene al nodo de mayor grado.

\subsection{Heurística de Búsqueda Local}

Arranco desde cualquier nodo o arista (este es mi $S_{0}$). Defino como vecindad a todos los posibles conjuntos de nodos que son una clique y que cumplen lo siguiente:
\begin{itemize}
\item Tiene todos los nodos de $S_{0}$ salvo 1.
\item Tiene todos los nodos de $S_{0}$ mas uno que no pertenecia a $S_{0}$ y que forma una clique con los demas 
\item Tiene todos los nodos de $S_{0}$ salvo 1 que se lo reemplaza por uno q no estaba en $S_{0}$. Esto significa que agrego un nodo y quito otro (al mismo tiempo).
\end{itemize}

de todo este conjunto tomo el que tiene mayor frontera (si es que esta es mayor a la frontera de $S_{0}$)

%no entendi esto! 
%Hay que ver cuánto dura cada vértice en cantidad de ejecuciones.


\subsection{Heurística de Búsqueda Tabú}
