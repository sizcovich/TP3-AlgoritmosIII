En este ejercicio, nos limitamos a pensar situaciones de la vida real que pudieran ser modeladas con $cliques\ de\ máxima\ frontera$.
\begin{itemize}

\item La conocida empresa internacional \textit{Chumbawamba} decide formar un equipo de trabajo para un proyecto que cambiará el mundo. Para que éste funcione sin peleas de por medio, dicho equipo debe consistir en un grupo de personas que se conozcan entre sí. Dado que la empresa desea que el proyecto sea conocido por la mayor cantidad de gente posible (siendo el boca en boca su única forma de difusión), deciden seleccionar a sus empleados de acuerdo a un conjunto que se conozca entre sí y que, también, conozca a la mayor cantidad de gente posible fuera de la empresa. Para modelar este problema, se consideran a los nodos como personas y a las aristas como la representación de un vínculo existente entre una persona y otra.

\item Un conocido golpeador desea asistir a la mejor fiesta todos los tiempos. Debido a su tardío horario de llegada al evento, éste se percata de que no puede bailar tranquilamente en el espacio reducido que encontró. Bajo un ataque de furia decide comenzar a pegarle a la gente de forma estratégica para que, al tirar a una persona, se produzca un efecto dominó y se caigan todas las que se encuentran a menos de treinta centímetros de ésta. De este modo, lograría tirar la mayor cantidad de gente al piso y disfrutar de la fiesta tranquilo. Luego, nota que le conviene comenzar a golpear a las personas que se encuentran a menos de un metro y medio de distancia entre sí y que, además, se encuentran a menos de dicha distancia con la mayor cantidad de personas para que el alcance sea mayor. Para modelar este problema, consideramos a los nodos como personas y a las arista como la representación de que la distancia entre una persona y otra es menor a un metro y medio.  

\item En un campo casi como cualquier otro, se encuentran plantas que deben ser regadas con gran potencia para no morir. Debido a que, a medida que el agua atraviesa una planta, ésta pierde potencia, el granjero \textit{Don Omar} busca que la mayor cantidad de plantas sean regadas con la mayor potencia posible. Dado que el granjero es una persona de gran capacidad pero no posee un enorme presupuesto, decide armar una estrategia de riego para sus amadas plantas teniendo que seleccionar un único circuito. Como cuenta con un único circuito de riego en el que Ésta consiste en regar alguna planta dentro de un cí de para que se muera la menor cantidad de plantas. Para modelar este problema, se consideran a los nodos como plantas y a las aristas como caminos del sistema de riego por donde puede circular el agua. %no esta terminado

\item Juanita es una apasionada de conocer calles nuevas y desea mudarse a una ciudad que le permita seguir con su pasión. Pero Juanita trabaja en el campo, o sea que fuera de cualquier ciudad, así que se le ocurre seleccionar su próxima vivienda de acuerdo a la cantidad de calles que va a poder conocer para llegar al campo. Podemos modelar el problema como los nodos representantes de las esquinas y las aristas las interconexiones entre éstas. Llamamos ciudad a todas las esquinas que se conectan todas con todas, luego, a una clique y campo a todo lo que se encuentra entre una ciudad y otra. Por lo tanto, se desea encontrar una ciudad con mayor frontera para que la cantidad de posibles calles nuevas a recorrer fuera de la ciudad sea la mayor.

\end{itemize}
