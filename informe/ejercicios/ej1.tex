En este ejercicio, nos limitamos a pensar situaciones de la vida real que pudieran ser modeladas con $cliques\ de\ máxima\ frontera$.
\begin{itemize}
\item En primer lugar, se podría modelar a un conjunto de casas como un grafo. Cada vértice representaría a una casa y cada arista a un camino que los une. Supongamos que dichos pueblos requieren de salud pública pero el presupuesto alcanza únicamente para la construcción de un hospital. Luego, lo más razonable consiste en ubicar el hospital en el pueblo con mayor cantidad de casas capaces de acceder al hospital para que sean más las personas que puedan alcanzarlo.


\item Se quiere elegir un elemento. propagar un virus de la forma más rápida donde cada nodo representa un elemento infeccioso. Los elementos se propagan a traves de las aristas y cuando uno de estos se infecta, éste infecta a sus nodos adyacentes. Luego, se busca un nodo dentro del conjunto de nodos que se interconectan entre si y que se conectan con la mayor cantidad de elementos de afuera. 

\item Equipo de trabajo: Voy a elegir a la mayor cantidad de personas que se conozcan entre sí  

\item Virus distruibuido, quiero infectar la mayor cantidad de máquinas posible.
 
\end{itemize}
