En este ejercicio, nos limitamos a pensar situaciones de la vida real que pudieran ser modeladas con $cliques\ de\ máxima\ frontera$.
\begin{itemize}
\item Juanita es una apasionada de conocer calles nuevas y desea mudarse a algún barrio de una ciudad que le permita seguir con su pasión. Pero Juanita trabaja en el campo, o sea que fuera de cualquier ciudad, así que se le ocurre seleccionar su próxima vivienda de acuerdo a la cantidad de calles que va a poder conocer para llegar al campo. Podemos modelar el problema como los nodos representantes de las esquinas y las aristas las interconexiones entre éstas. Llamamos ciudad a todas las esquinas que se conectan todas con todas, luego, a una clique y campo a todo lo que se encuentra entre una ciudad y otra. Por lo tanto, se desea encontrar una ciudad con mayor frontera para que la cantidad de posibles calles nuevas a recorrer fuera de la ciudad sea la mayor.

\item Se vienen las elecciones y como todos los años, el futuro candidato a presidente Josecito quiere invertir en repavimentación. Esta vez quiere elegir las esquinas interconectadas entre sí que tengan la mayor cantidad de calles en su frontera. Esto se debe a que las máquinas de repavimentación, que se encuentran en cada esquina de la zona de trabajo, necesitan estar conectadas entre sí dado que, cuando se quedan sin algún material necesitan llegar rápidamente a otra de las máquinas para terminar velozmente el trabajo. Si modelamos a las esquinas como nodos y a las aristas como calles, Josecito debe encontrar la clique de mayor frontera. 
\end{itemize}
