En este ejercicio, nos limitamos a pensar situaciones de la vida real que pudieran ser modeladas con $cliques\ de\ máxima\ frontera$.
\begin{itemize}
\item En primer lugar, se podría modelar a un conjunto de casas como un grafo. Cada vértice representaría a una casa y cada arista a un camino que los une. Supongamos que dichos pueblos requieren de salud pública pero el presupuesto alcanza únicamente para la construcción de un hospital. Luego, lo más razonable consiste en ubicar el hospital en el pueblo con mayor cantidad de casas capaces de acceder al hospital para que sean más las personas que puedan alcanzarlo.

\item Organización de más beneficiencia

\item Amigos de Facebook

\item Una máquina de zombies
 
\end{itemize}