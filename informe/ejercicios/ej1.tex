En este ejercicio, nos limitamos a pensar situaciones de la vida real que pudieran ser modeladas con $cliques\ de\ máxima\ frontera$.
\begin{itemize}
\item En primer lugar, se podría modelar a un conjunto de pueblos como un grafo. Cada vértice representaría a un pueblo y cada arista a un camino que los une. Supongamos que dichos pueblos requieren de salud pública pero el presupuesto alcanza únicamente para la construcción de un hospital. Luego, lo más razonable consiste en ubicar el hospital en el pueblo con mayor cantidad de accesos para que sean más los pueblos que puedan alcanzarlo.

\item 
 
\end{itemize}