\subsection{Heurística de Búsqueda Tabú}

\subsubsection{Explicación del algoritmo realizado}
\subsubsection{Complejidad Temporal}
\subsubsection{Instancias problemáticas}
\subsubsection{Experimentación}


\begin{algorithm}[H]
    \SetAlgoLined
    \caption{TabuSearch}
    \KwIn{\textbf{Conj(nodos)} $solución\_inicial$, \textbf{Grafo} $g$, \textbf{Entero} $cantidad\_pasos$, \textbf{Entero} $desviacion\_permitida$}
    \KwOut{\textbf{Conj(nodos)} $solución\_final$}
		
	\textbf{Entero} $desviacion\_permitida\_aux$ = 0 \\
	\textbf{Conj(nodos)} $solución\_final$ = $solución\_inicial$ \\
		
	\While{$cantidad\_pasos >$ 0}{
		\textbf{Entero} $frontera\_ini$ = frontera($solución\_inicial$) \\
	 	\For{$u \in Candidatos\_clique(g,u)$}{
	 		\If{$\neg$ es tabu($u$) $\land \neg$ esta agregado $u$ en $solución\_inicial$}{
	 			\eIf{frontera($solución\_final$) $<$ Frontera con $u$ en $solución\_inicial$}{
					$solución\_final$ = $solución\_inicial$ con $u$}{
	  				\If{$desviacion\_permitida\_aux >$ 0}{
	 				$solución\_inicial$ = $solución\_inicial$ con $u$ \\
	 				Poner en lista Tabu a $u$ \\
	 				$desviacion\_permitida\_aux$ - 1}}}}}
	
	\ForAll{$u \in$ Nodos($solución$\_$inicial$)}{
		\If{$\neg$ es tabu($u$)}{
	 		\eIf{frontera($solución\_final$) $<$ Frontera sin $u$ en $solución\_inicial$}{
	 			$solución\_final$ = $solución\_inicial$ sin $u$}{
				\If{$desviacion\_permitida\_aux >$ 0}{
					$solución\_inicial$ = $solución\_inicial$ sin $u$ \\
					Poner en lista Tabu a $u$ \\
					$desviacion\_permitida\_aux$ - 1}}}}
							
	\If{$desviacion\_permitida\_aux \leq$ 0}{
		$solución\_inicial$ = $solución\_final$}
	\While{$frontera\_ini <$ frontera($solución\_final$) $\vee \ desviacion\_permitida\_aux >$ 0}{
		$desviacion\_permitida\_aux$ = $desviacion\_permitida$ \\
		Vaciar la lista Tabu \\
		Agregar los ultimos dos nodos de $solución\_final$ a la lista Tabu \\
		$cantidad\_pasos$ - 1}
    	\textbf{devolver} $solución\_final$ \\

\end{algorithm}

Donde:
\begin{itemize}
 \item $desviacion$\_$permitida$ dice la cantidad de veces que se agrega o quita un nodo por iteracion (empeorando la solución parcial).
 \item $cantidad$\_$pasos$ son la cantidad de veces que se aplica el algoritmo. Tener en cuenta que la primera iteracion $desviacion$\_$permitida$ es 0, por lo que se toma el maximo local.
 \item frontera : Dice, dada una solución como parametro, el numero de nodos adyacentes a la frontera (es lo que pide maximizar el enunciado).
 \item Candidatos$\_$clique : Dice los nodos que pertenecen a la clique del nodo pasado como parametro.
 \item Nodos : da los nodos pertenecientes a la solución pasada como parametro.
\end{itemize}
